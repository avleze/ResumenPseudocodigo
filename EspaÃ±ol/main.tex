\documentclass{article}
\usepackage{multicol}
\usepackage{amsfonts}

\usepackage{polyglossia}
\setmainlanguage{spanish}



\usepackage{fancyhdr}
\renewcommand{\headrulewidth}{0.4pt}
\renewcommand{\footrulewidth}{0.4pt}
\rhead{\small\emph{\nouppercase{\rightmark}}}
\lhead{\small\emph{\nouppercase{\leftmark}}}
\usepackage[titletoc]{appendix}
\usepackage[dvipsnames]{xcolor}
\usepackage[a4paper,width=130mm,top=25mm,bottom=25mm,bindingoffset=6mm]{geometry}
\usepackage[hang,flushmargin]{footmisc}
\usepackage{listings}
\definecolor{azulpseudo}{HTML}{1919FF}
\definecolor{secnaranja}{HTML}{FFA500}
\definecolor{ccomentarios}{HTML}{009900}
\newcommand{\pkeyword}[1]{\textcolor{azulpseudo}{\texttt{\textbf{#1}}}}
% Colores
\definecolor{azulpseudo}{HTML}{1919FF}
\definecolor{secnaranja}{HTML}{FFA500}
\definecolor{ccomentarios}{HTML}{009900}

\lstdefinelanguage{pseudocodigoesp}
                  {
                    morecomment= [l]{//},
                   morecomment= [s]{/*}{*/},
                   morestring=[b]",
                    commentstyle = \color{ccomentarios},
                    keywords =
                    { Algoritmo,
                      fin_algoritmo,
                      inicio,
                      var,
                      tipo,
                      principal,
                      fin_principal,
                      div,
                      mientras,
                      hacer,
                      fin_mientras,
                      repetir,
                      hasta_que,
                      desde,
                      hasta,
                      fin_desde,
                      si,
                      si_no,
                      fin_si,
                      segun_sea,
                      fin_segun,
                      en_otro_caso,
                      entonces,
                      vector,
                      de,
                      matriz,
                      entero,
                      real,
                      logico,
                      caracter,
                      cadena,
                      funcion,
                      fin_funcion,
                      devolver,
                      procedimiento,
                      fin_procedimiento,
                      registro,
                      fin_registro,
                      escribir,
                      leer,
                      archivo,
                      concatena,
                      abrir,
                      cerrar,
                      feof,
                      const,
                      tipos},
                      morecomment= [l]{//},
                      morecomment= [s]{/*}{*/},
                      literate={
                      {<-}{{\textcolor{red}{{$\leftarrow$}}}}2
                      {<=}{{$\textcolor{red}{\leq}$}}1
                      {>=}{{$\textcolor{red}{\geq}$}}1
                      {!=}{{$\textcolor{red}{\neq}$}}1
                      {+}{{\textcolor{red}{+}}}1
                      {-}{{\textcolor{red}{-}}}1
                    %  {*}{{\textcolor{red}{*}}}1
                      {=}{{$\textcolor{red}{=}$}}2
                      {E real:}{{$\texttt{\textcolor{blue}{E entero:}}$}}9
                      {entero:}{{$\texttt{\textcolor{blue}{entero:}}$}}6
                      {logico:}{{$\texttt{\textcolor{blue}{logico:}}$}}6
                      {real:}{{$\texttt{\textcolor{blue}{real:}}$}}3
                      {caracter:}{{$\texttt{\textcolor{blue}{caracter:}}$}}8
                      {\ \ }{{\ }}1}
                      ,
                   morestring=[b]",
                   showstringspaces=false,
                    commentstyle = \color{ccomentarios},
                  }
                  \lstset{
                    basicstyle = {\small\ttfamily},
                    keywordstyle = \color{blue}\textbf,
                    stringstyle = \ttfamily\color{ccomentarios},
                    inputencoding = utf8,
                    extendedchars = true,
                  }

%Opciones generales del documento
\setlength\parindent{0pt}
\setlength\parskip{5pt}
\usepackage{mflogo}
\usepackage[bookmarksopen=true]{hyperref}

\title{Resumen Pseudocódigo}
\author{Antonio Vélez Estévez}
\date{Julio 2016}
\pagestyle{fancy}
\begin{document}

\maketitle


\begin{abstract}
Este documento es un pequeño manual-resumen del pseudocódigo usado en la asignatura de Introducción a la Programación en la Universidad de Cádiz.
\end{abstract}
\tableofcontents
\pagebreak
\section{Vista general}
La estructura general de un algoritmo en pseudocódigo es la siguiente:

\begin{lstlisting}[language=pseudocodigoesp]
Algoritmo nombreAlgoritmo
    const
        //sección de definición de constantes.
    tipos
        //sección de definición de tipos.
    var
        //sección de definición de variables globales.


        //definición de funciones y procedimientos.
principal
    var
        // sección de definición de variables locales a Principal.
    inicio
        // inicialización de variables...
        // instrucciones del algoritmo principal...
fin_principal

fin_algoritmo
\end{lstlisting}

\section{Tipos de datos}
En pseudocódigo disponemos de los tipos de datos simples y estructurados para diseñar nuestros programas.
\subsection{Simples}
\subsubsection{Numéricos}
\paragraph{Enteros} es un subconjunto finito de los enteros ($\mathbb{Z}$), el rango disponible es: $$\left[-32768, 32768\right]$$

Para declarar una variable de tipo entero en pseudocódigo usaremos la palabra reservada \pkeyword{entero}.

\paragraph{Reales} es un subconjunto finito de los reales ($\mathbb{R}$), el rango disponible es: $$\left[ 1,17549\times 10^{-38}, 3,40282\times 10^{38} \right]$$


Para declarar una variable de tipo real en pseudocódigo usaremos la palabra reservada \pkeyword{real}.

\subsubsection{Lógicos}
Este tipo de dato solo puede tener dos valores \textbf{verdadero} o \textbf{falso}, sobre él se pueden aplicar los operadores del algebra booleana
\footnote{Conjunción, disyunción, negación y disyunción excluyente (habitualmente conocida como \textbf{XOR} de ``eXclusive OR'' ).}.

Para declarar una variable de este tipo en pseudocódigo usaremos la palabra reservada \pkeyword{logico}.

\subsubsection{Carácter}

Puede representar los carácteres alfanuméricos especificados en el código ASCII\footnote{American Standart Code for Information Interchange.
  El código ASCII utiliza 7 bits para representar los carácteres.}.

Para declarar una variable de tipo carácter en pseudocódigo usaremos la palabra reservada \pkeyword{caracter}.

\subsection{Estructurados}
Los tipos de datos estructurados se definirán en la sección de definición de tipos, que está encabezada por la palabra reservada \pkeyword{tipos}.
\subsubsection{Vectores}
Un vector es un conjunto finito de elementos \textit{del mismo tipo}, que están almacenados consecutivamente y que pueden ser identificados de forma independiente.

Para definir un vector en la sección de tipos usaremos la siguiente sintaxis:

\begin{lstlisting}[language = pseudocodigoesp]
vector [tamaño] de <tipo_dato> : <identificador_tipo_vector>  
\end{lstlisting}

Luego usaremos ese identificador para declarar el tipo de una variable.

\textbf{Nota\footnote{En Vary debido a temas de diseño los vectores empiezan en la posición 0 al igual que en el lenguaje C.}:} los vectores en pseudocódigo empiezan en la posición 1.
\subsubsection{Matrices}
Una matriz multidimensional es una estructura homogénea, en la cual para hacer referencia a un elemento necesitamos dos o más índices dependiendo de su dimensión.

Para definir una matriz en la sección de tipos usaremos la siguiente sintaxis:
\begin{lstlisting}[language = pseudocodigoesp]
matriz [tam1, ..., tamN] de <tipo_dato> : <identificador_tipo_matriz>
\end{lstlisting}

Para declarar una variable de dicho tipo usaremos el identificador que establezcamos.
\subsubsection{Cadenas de caracteres}
Una cadena de caracteres es una secuencia de caracteres consecutivos. Para declarar una variable como cadena debemos usar la palabra clave \pkeyword{cadena}.
\begin{itemize}
\item Se puede asignar a una variable de tipo \pkeyword{cadena} una constante de cadena.
\item Se pueden usar las operaciones de lectura y escritura con cadenas de caracteres.
\item Podemos calcular la longitud de una cadena con \pkeyword{longitud}(\#1).
\item Se pueden comparar cadenas con los operadores relacionales conocidos\footnote{menor-que(<), mayor-que(>), igual(=), menor o igual ($\leq$), mayor o igual ($\geq$) y distinto ($\neq$).}.
\item Podemos concatenar dos cadenas con \pkeyword{concatena}(\#1, \#2)
\end{itemize}

\subsubsection{Registros}
Es una estructura de datos formada por un conjunto de elementos que contienen información relativa a algo. Los elementos que constituyen un \pkeyword{registro} se denominan \emph{campos} y cada campo puede ser de un tipo diferente.
\pagebreak

Para definir un \pkeyword{registro} en la sección de tipos usaremos la siguiente sintaxis:

\begin{lstlisting}[language = pseudocodigoesp]
registro : nombreRegistro
    tipo1 : idCampo1
    tipo2 : idCampo2
          .
    tipoN : idCampoN 
fin_registro
\end{lstlisting}

Consideraciones:

\begin{itemize}
\item Se puede realizar la asignación entre dos registros completos siempre que sean del mismo tipo.
\end{itemize}
\subsubsection{Ficheros}
Para definir un tipo de \pkeyword{fichero} en la sección de tipos usaremos la siguiente sintaxis:
\begin{lstlisting}[language = pseudocodigoesp]
archivo de tipo_dato : tipoFichero
\end{lstlisting}

Para abrir un fichero usaremos la funcion \pkeyword{abrir}() que asocia una variable de tipo fichero con un archivo. El formato de dicha función es el siguiente:
\begin{lstlisting}[language = pseudocodigoesp]
abrir(varFichero, modo, nombreFichero)
\end{lstlisting}
Donde:
\begin{itemize}
\item\textbf{\texttt{varFichero}} es la variable, declarada previamente a la que se le asociará el fichero físico.
\item\textbf{\texttt{modo}} indica el modo de acceso al fichero(escritura o lectura).
\item\textbf{\texttt{nombreFichero}} es el nombre del fichero que se encuentra almacenado en memoria masiva.
\end{itemize}

Para el manejo de ficheros disponemos de cuatro operaciones:

\begin{itemize}
\item \pkeyword{escribir}\texttt{(varFichero, elemento)}.
\item \pkeyword{leer}\texttt{(varFichero, elemento)}.
\item \pkeyword{feof}\texttt{(varFichero)}.
\item \pkeyword{cerrar}\texttt{(varFichero)}.
\end{itemize}
\subsubsection{Enumerados}
El uso de tipos enumerados permite mejorar la legibilidad de los algoritmos y reducir posibles errores. Para definir un tipo enumerado usaremos la siguiente sintaxis en la sección de tipos:
\begin{lstlisting}[language = pseudocodigoesp]
identificador_tipo_enumerado = {elemento1, ..., elementoN}
\end{lstlisting}

Luego para declarar una variable como este tipo enumerado usaremos dicho identificador.

\subsubsection{Subrango}
Contiene un rango de valores de otro tipo existente, sea predefinido o definido por el usuario. Para definir un subrango usaremos la siguiente sintaxis en la sección de tipos:
\begin{lstlisting}[language = pseudocodigoesp]
identificador_tipo_subrango = limInf .. limSup
\end{lstlisting}

Para declarar una variable como este tipo subrango usaremos dicho identificador.
\section{Variables, constantes y expresiones}
Una \textbf{variable} es un objeto que contiene un valor que puede variar durante la ejecución del programa, mientras que una \textbf{constante} también es un objeto que contiene un valor, pero que, a diferencia de una variable, su valor no cambia.

Para usar una variable primero tenemos que declararla, esto es, asociar un tipo con la misma. En pseudocódigo esto se hace en la sección \pkeyword{var} de la siguiente forma:
\begin{lstlisting}[language = pseudocodigoesp]
identificador_del_tipo : identificador_de_la_variable
\end{lstlisting}

Para usar una constante primero tenemos que definirla, esto es darle un valor. En pseudocódigo esto se hace en la sección \pkeyword{const} de la siguiente forma:
\begin{lstlisting}[language = pseudocodigoesp]
identificador_de_la_constante = valor_de_la_constante
\end{lstlisting}
\subsection{Asignación}
Se utiliza para darle valor a una variable y se representa por $\leftarrow$, el formato es:
\begin{lstlisting}[language = pseudocodigoesp]
identificador_de_variable <- expresión
\end{lstlisting}

Esta operación es destructiva, el valor que tenía antes la variable desaparece.
\subsection{Expresiones aritméticas}
Las expresiones aritméticas que se pueden realizar vienen resumidas en la siguiente tabla:
\begin{center}
  \begin{tabular}{| c | l | c | l}
    + & Suma & \texttt{|} & OR a nivel de bits\\[8pt]
    - & Resta & \texttt{\&} & AND a nivel de bits\\[8pt]
    * & Producto & \pkeyword{xor} & XOR a nivel de bits\\[8pt]
    / & División entera & &\\[8pt]
    \pkeyword{div} & División real & &
  \end{tabular} 
\end{center}
\subsection{Expresiones lógicas}
Las expresiones lógicas que se pueden realizar vienen resumidas den la siguiente tabla:

\begin{center}
  \begin{tabular}{| c | l |}
    \pkeyword{no} ó \texttt{!} & Negación lógica\\[8pt]
    \pkeyword{y} ó \texttt{\&\&} & Conjunción\\[8pt]
    \pkeyword{o} ó \texttt{||} & Disyunción\\[8pt]
    \texttt{=} & Igualdad\\[8pt]
    \texttt{!=} ó \texttt{<>} & Desigualdad\\[8pt]
    \texttt{$\geq$} ó \texttt{>=} & Mayor o igual \\[8pt]
    \texttt{$\leq$} ó \texttt{<=} & Menor o igual \\[8pt]
    \texttt{<} & Menor que \\[8pt]
    \texttt{>} & Mayor que \\[8pt]
  \end{tabular} 
\end{center}

\section{Estructuras de control}
\subsection{Selectivas}
Permiten decidir entre varios bloques de código en base a una condición.
\subsubsection{Simple}
La estructura general de esta estructura selectiva es:
\begin{lstlisting}[language = pseudocodigoesp]
si (condición) entonces
    // Bloque que se ejecuta únicamente si la condición
    // es verdadera.
fin_si
\end{lstlisting}
\subsubsection{Doble}
La estructura general de esta estructura selectiva es:
\begin{lstlisting}[language = pseudocodigoesp]
si (condición) entonces
    // Bloque que se ejecuta únicamente si la condición
    // es verdadera.
si_no
    // Bloque que se ejecuta únicamente si la condición
    // es falsa.
fin_si
\end{lstlisting}
\subsubsection{Múltiple}
La estructura general de esta estructura selectiva es:
\begin{lstlisting}[language = pseudocodigoesp]
segun_sea (expresión) hacer
    1:
        // Bloque de código si la expresión resultante es 1.
    2:
        // Bloque de código si la expresión resultante es 2.
    .
    .
    .
    n:
        // Bloque de código si la expresión resultante es n.
    en_otro_caso:
        // Bloque de código en un caso contrario a los anteriores.
fin_segun
\end{lstlisting}
\subsection{Repetitivas}
Permiten repetir una o varias instrucciones varias veces en función
de la evaluación de una determinada condición. A las estructuras repetitivas
se les denomina \emph{bucles}, y se llama \emph{iteración} a cada repetición de la
ejecución de la secuencia de instrucciones que forman el llamado
\emph{cuerpo del bucle}. 
\subsubsection{Mientras}
La plantilla de esta estructura repetitiva es:
\begin{lstlisting}[language = pseudocodigoesp]
mientras (condición) hacer
    // cuerpo del bucle
fin_mientras
\end{lstlisting}
\subsubsection{Repetir}
La plantilla de esta estructura repetitiva es:
\begin{lstlisting}[language = pseudocodigoesp]
repetir
    // cuerpo del bucle
hasta_que (condición)
\end{lstlisting}

Esta estructura tiene una peculiaridad, \textbf{se ejecuta mientras la condición se evalúa como falsa}, cuando es verdadera termina su ejecución.
\subsubsection{Desde}
La plantilla de esta estructura repetitiva es:
\begin{lstlisting}[language = pseudocodigoesp, mathescape=true]
desde i <- $V_i$ hasta $V_f$ hacer
    // cuerpo del bucle
fin_desde
\end{lstlisting}

Por defecto el bucle \pkeyword{desde} aumenta la variable de control en una unidad en cada iteración, podemos cambiar esto poniendo:

\begin{lstlisting}[language = pseudocodigoesp, mathescape=true]
desde i <- $V_i$ hasta $V_f$ hacer i <- i + n
    // cuerpo del bucle
fin_desde
\end{lstlisting}

Siendo \texttt{n} el valor en el que se quiere incrementar o decrementar \texttt{i} en cada iteración.

\section{Subalgoritmos}
\subsection{Funciones}
Para declarar funciones en pseudocódigo usaremos la siguiente plantilla:
\begin{lstlisting}[language = pseudocodigoesp]
<especificación_de_la_función>
<tipo_del_resultado> funcion <nombre_función> (<lista_de_parámetros_formales>)
    var
    // Sección de definición de variables locales a la función,
    // esta sección es opcional.
    inicio
    // cuerpo de la función
    devolver (<expresión_resultado>)
fin_funcion
\end{lstlisting}
Donde:
\begin{itemize}
\item \texttt{lista\_de\_parámetros\_formales} es una lista de la siguiente forma:\linebreak\newline\texttt{(\pkeyword{\{E|S|E/S\} tipo1}: param1,{\color{red}...}, \pkeyword{\{E|S|E/S\} tipoN}: paramN)}
  \begin{itemize} 
    \item El significado de \pkeyword{E}, \pkeyword{S} y \pkeyword{E/S} se detallará en \ref{subsec:pvr}\nameref{subsec:pvr}.
    \end{itemize}
\end{itemize}
\pagebreak
\subsection{Procedimientos}
Para declarar procedimientos en pseudocódigo usaremos la siguiente plantilla:
\begin{lstlisting}[language = pseudocodigoesp]
<especificación_del_procedimiento>
procedimiento <nombre_procedimiento> (<lista_de_parámetros_formales>) 
    var
    // Sección de definición de variables locales al procedimiento,
    // esta sección es opcional.
    inicio
    // cuerpo del procedimiento.
fin_procedimiento
\end{lstlisting}
La lista de parámetros formales sigue la misma forma que en la sección anterior. Se puede observar como no hay \pkeyword{devolver}, ya que los procedimientos no devuelven nada.
\subsection{Paso por valor y por referencia}
\label{subsec:pvr}
En el paso por valor se produce una copia del valor de los parámetros actuales en los parámetros formales. Este tipo de parámetros siempre será de entrada y llevarán en su declaración una \pkeyword{E}.

En el paso por referencia tenemos dos casos, en ambos el parámetro formal recibe una referencia del parámetro actual:
\begin{itemize}
\item \textbf{Salida} en este tipo solo podemos usar el parámetro formal en un valor izquierdo, es decir, en la parte izquierda de las asignaciones. Se declara poniendo \pkeyword{S}.
\item \textbf{Entrada/Salida} en este tipo podemos usar el parámetro formal tanto en un valor izquierdo como en uno derecho, es decir podemos usar su valor para calcular cualquier cosa y también podemos guardar valores en el mismo. Se declara poniendo \pkeyword{E/S}.
\end{itemize}

En el paso por valor los parámetros actuales \textbf{\textit{no se modifican}} mientras que en el paso por referencia \textbf{\textit{pueden modificarse}}.

\section{Ámbito de las variables}
En pseudocódigo tenemos dos tipos de ámbito:
\begin{itemize}
\item \textbf{Global}. Las variables son accesibles y visibles en el resto del código.
\item \textbf{Local}. Las variables locales son accesibles y visibles solamente en el bloque en el que se definen.
\end{itemize}

Para declarar una variable \textbf{global} en pseudocódigo debemos escribirla en la sección \pkeyword{var} del \pkeyword{Algoritmo}.

Para declarar una variable \textbf{local} a un bloque en pseudocódigo debemos escribirla en la sección \pkeyword{var} de dicho bloque.

\section{Funciones y procedimientos como parámetros}
Para pasar funciones o procedimientos como parámetros primero debemos declarar un tipo que sea dicha función o procedimiento siguiendo la siguiente sintaxis:

Para funciones:
\begin{lstlisting}[language = pseudocodigoesp]
[tipo_resultado] funcion (<lista_parametros_formales>) : id
\end{lstlisting}

Para procedimientos:
\begin{lstlisting}[language = pseudocodigoesp]
procedimiento (<lista_parametros_formales>) : id
\end{lstlisting}

Luego se usa dicho tipo como un tipo normal y corriente. Lo único especial que tienen las variables declaradas de este tipo es que pueden ser llamadas como una función.
\pagebreak
\begin{appendices}
  \section{Consideraciones en Vary}
  \pagebreak
  \section{Palabras reservadas}
  \pkeyword{
  \begin{tabular}{l l l l l }
    Algoritmo & fin\_algoritmo & inicio & var & tipo  \\[8pt]
    principal & fin\_principal & mientras & hacer & fin\_mientras \\[8pt]
    repetir & hasta\_que & desde & hasta & fin\_desde \\[8pt]
    si & si\_no & fin\_si & segun\_sea & en\_otro\_caso \\[8pt]
    entonces & vector & de &  matriz & entero \\[8pt]
    real & logico & caracter & cadena & funcion \\[8pt]
    fin\_funcion & devolver & procedimiento & fin\_procedimiento & registro\\[8pt]
    fin\_registro & escribir & leer & archivo & concatena \\[8pt]
    abrir & cerrar & feof & const & tipos \\[8pt]
    y & o & no \\
  \end{tabular}  }
   \pagebreak
   \section{Ejemplos}
   \subsection{Selectiva simple}
\begin{lstlisting}[language = pseudocodigoesp]
// Algoritmo: Ejemplo de estructura selectiva simple.
// Autor: Antonio Vélez Estévez
// Fecha: 23/04/2016
 
Algoritmo EstructurasSelectiva01

//Sección de definición de variables constantes
const 
	
//Sección de definición de tipos
tipo

//Sección de declaración de variables globales	
var

//Sección de definición de subalgoritmos: funciones y procedimientos	

//Comienzo del algoritmo (Obligatorio)
principal	
var
	// Definimos la variable condición de tipo lógico, es decir tomará
	// dos valores únicos, verdadero o falso.
	logico : condicion
inicio 	

	// Asignamos a la variable condicion el valor verdadero.
	condicion <- verdadero
	
	// Si el valor de la variable condición es verdadero, entonces
	// se ejecutará el código que hay entre el entonces y el fin_si.
	si (condicion = verdadero) entonces
		escribir("La condicion era verdadera!")
	fin_si
	
	// Asignamos a la variable condicion el valor falso.
	condicion <- falso
	
	// Si el valor de la variable condición es verdadero(nunca pasará
	// ya que acabamos de establecer la variable condicion a falso)
	// , entonces se ejecutará el código que hay entre el entonces y el fin_si.
	si (condicion = verdadero) entonces 
		escribir("Esto nunca se escribira!")
	fin_si
fin_principal

fin_algoritmo

\end{lstlisting}
\subsection{Selectiva doble}
\begin{lstlisting}[language = pseudocodigoesp]
// Algoritmo: Ejemplo de estructura selectiva doble.
// Autor: Antonio Vélez Estévez
// Fecha: 23/04/2016

Algoritmo EstructuraSelectiva02

//Sección de definición de variables constantes
const 
	
//Sección de definición de tipos
tipo

//Sección de declaración de variables globales	
var

//Sección de definición de subalgoritmos: funciones y procedimientos	

//Comienzo del algoritmo (Obligatorio)
principal	
var
	// Definimos la variable condición de tipo lógico, es decir tomará
	// dos valores únicos, verdadero o falso.
	logico : condicion
inicio 	

	// Asignamos a la variable condicion el valor verdadero.
	condicion <- verdadero
	
	// Si el valor de la variable condición es verdadero, entonces
	// se ejecutará el código que hay entre el entonces y el fin_si.
	// Si el valor de la variable es falso, entonces se ejecutará,
	// el código entre el si_no y el fin_si.
	si (condicion = verdadero) entonces
		escribir("La condicion era verdadera!")
	si_no
		escribir("Esto nunca se escribira!")
	fin_si

fin_principal
fin_algoritmo
\end{lstlisting}
\subsection{Selectiva múltiple}
\begin{lstlisting}[language = pseudocodigoesp]
// Algoritmo: Ejemplo de estructura selectiva múltiple.
// Autor: Antonio Vélez Estévez
// Fecha: 23/04/2016
 
Algoritmo EstructurasSelectiva03

//Sección de definición de variables constantes
const 
	
//Sección de definición de tipos
tipo

//Sección de declaración de variables globales	
var

//Sección de definición de subalgoritmos: funciones y procedimientos	

//Comienzo del algoritmo (Obligatorio)
principal	
var
	// Definimos la variable condición de tipo entero, es decir tomará
	// valores en dicho conjunto.
	entero : numero
inicio 	

	// Le muestra al usuario el siguiente mensaje por la pantalla
	escribir("Ingrese un numero de un solo digito: ")
	// Lee un valor y lo guarda en la variable numero.
	leer(numero)
	
	
	// Segun sea el numero, en el caso de que sea uno, escribira
	// Su numero es el uno.
	segun_sea(numero) hacer
		caso 1: escribir("Su numero es el uno.")
		caso 2: escribir("Su numero es el dos.")
		caso 3: escribir("Su numero es el tres.")
		caso 4: escribir("Su numero es el cuatro.")
		caso 5: escribir("Su numero es el cinco.")
		caso 6: escribir("Su numero es el seis.")
		caso 7: escribir("Su numero es el siete.")
		caso 8: escribir("Su numero es el ocho.")
		caso 9: escribir("Su numero es el nueve.")
		caso 0: escribir("Su numero es el cero.")
		en_otro_caso: escribir("No se cual es su numero, probablemente
                haya introducido algo fuera del rango establecido.")
	fin_segun

fin_principal
fin_algoritmo
\end{lstlisting}
\subsection{Bucle mientras}
\begin{lstlisting}[language = pseudocodigoesp]

// Algoritmo: Ejemplo de la estructura repetitiva mientras.
// Autor: Antonio Vélez Estévez
// Fecha: 23/04/2016

 
Algoritmo EstructurasRepetitiva01

//Sección de definición de variables constantes
const 
	
//Sección de definición de tipos
tipo

//Sección de declaración de variables globales	
var

//Sección de definición de subalgoritmos: funciones y procedimientos	

//Comienzo del algoritmo (Obligatorio)
principal	
var
	// Definimos la variable condición de tipo entero, es decir tomará
	// valores en dicho conjunto.
	entero : numero
inicio 	
	// Le muestra al usuario el siguiente mensaje por la pantalla
	escribir("Ingrese un numero de un solo digito: ")
	// Lee un valor y lo guarda en la variable numero.
	leer(numero)
	
	// Le muestra al usuario el siguiente mensaje por la pantalla
	escribir("Ahora le mostrare todos los numeros que hay desde el
        numero que ha introducido hasta el 1000")
	
	// Ahora tenemos el número que ha introducido el usuario,
	// imaginemos que es el 992, como el 992 es menor que el
	// 1000, se ejecutará el cuerpo del mientras, se escribirá
	// el numero 992, y se incrementará en uno la variable número.
	
	// La última iteración será cuando el valor de numero sea
	// mil, mil es menor o igual que mil, así que entraríamos
	// en el cuerpo del mientras, incrementaríamos en uno, y 
	// tendríamos 1001, al intentar entrar otra vez en el bucle
	// tenemos que comparar si 1001 es menor o igual que 1000,
	// cosa que es falsa, así que se sale del bucle y termina.
	mientras (numero <= 1000) hacer
		escribir("Voy por el numero ", numero)
		numero <- numero + 1
	fin_mientras
	

fin_principal
fin_algoritmo
\end{lstlisting}
\subsection{Bucle repetir}
\begin{lstlisting}[language = pseudocodigoesp]

// Algoritmo: Ejemplo de la estructura repetitiva repetir. Algoritmo
// para hacer un menu.
// Autor: Antonio Vélez Estévez
// Fecha: 23/04/2016
  
Algoritmo EstructurasRepetitiva02

//Sección de definición de variables constantes
const 
	
//Sección de definición de tipos
tipo

//Sección de declaración de variables globales	
var

//Sección de definición de subalgoritmos: funciones y procedimientos	

//Comienzo del algoritmo (Obligatorio)
principal	
var
	// Definimos la variable opcion de tipo entero, es decir tomará
	// valores en dicho conjunto.
	entero : opcion
inicio 	
	// Esta estructura es muy útil para hacer menús
	// porque al menos se repiten una vez, ya que no
	// comprueban la condición al principio.
	
	repetir
		escribir("Opcion 1: Volar.")
		escribir("Opcion 2: Caminar.")
		escribir("Opcion 3: Nadar.")
		escribir("Opcion 4: Salir.")
		escribir("\n Elija una: ")
		leer(opcion)
		// Seguirá repitiendo el cuerpo, hasta que opción sea cuatro.
	hasta_que (opcion = 4)
	
	escribir("Usted ha decidido salir")

fin_principal
fin_algoritmo
  \end{lstlisting}
\subsection{Bucle for}
\begin{lstlisting}[language = pseudocodigoesp]
// Algoritmo: Ejemplo de la estructura repetitiva desde.
// Autor: Antonio Vélez Estévez
// Fecha: 23/04/2016
 
Algoritmo EstructurasRepetitiva03

//Sección de definición de variables constantes
const 
	
//Sección de definición de tipos
tipo

//Sección de declaración de variables globales	
var

//Sección de definición de subalgoritmos: funciones y procedimientos	

//Comienzo del algoritmo (Obligatorio)
principal	
var
	// Límite inferior del intervalo.
	entero: limInf
	// Límite superior del intervalo.
	entero: limSup
	// Variable de control del bucle.
	entero: z
	// Acumulador.
	entero: acumulador
inicio 	
	// Inicializamos el acumulador a cero.
	acumulador <- 0
	
	escribir("Introduzca el límite inferior del intervalo:")
	leer(limInf)
	escribir("\nIntroduzca el límite superior del intervalo")
	leer(limSup)
	
	// Algoritmo para sumar todos los números
	// en los límites del intervalo [i,j] que
	// introduzca el usuario.
	
	desde z <- limInf hasta limSup hacer
		acumulador <- acumulador + z
	fin_desde

	escribir("El resultado de la suma es: ", acumulador)
fin_principal
fin_algoritmo
\end{lstlisting}
\subsection{Funciones}
\subsubsection{Paso por valor}
\begin{lstlisting}[language = pseudocodigoesp, keepspaces]
// Algoritmo: Ejemplo de funcion con paso por valor
// Autor: Antonio Vélez Estévez
// Fecha: 30/06/2016
 
Algoritmo FuncionesPasoValor01

//Sección de definición de variables constantes
const 
//Sección de definición de tipos
tipo
//Sección de declaración de variables globales	
var

//Sección de definición de subalgoritmos: funciones y procedimientos	
real funcion elevar(E real: x, E real: n)
var
	entero: resultado
	entero: i
inicio
	resultado <- 1
	
	// Si es igual a 0 pasaría del bloque de código y devolvería directamente
        // el resultado.
	si(n != 0) entonces
		si(n > 0) entonces
			desde i <- 1 hasta n hacer
				resultado <- resultado * x
			fin_desde
		si_no
			desde i <- 1 hasta -n hacer
				resultado <- resultado / x
			fin_desde
		fin_si
	fin_si
	
	devolver resultado
fin_funcion
//Comienzo del algoritmo (Obligatorio)
principal	

var
         real: base
        entero: exponente
         real: resultado
inicio 	
	escribir("Introduzca un número: ")
	leer(base)
	
	escribir("Introduzca un exponente: ")
	leer(exponente)

	escribir("El resultado de la potencia es:", elevar(base, exponente))
fin_principal
fin_algoritmo
\end{lstlisting}
\subsubsection{Paso por referencia}
\begin{lstlisting}[language = pseudocodigoesp]
// Algoritmo: Ejemplo de función con paso por referencia.
// Realizar una función que dados los parámetros, vector por entrada y un real p
// or referencia, devuelva la media de los elementos del vector y guarde en el s
// egundo parámetro la suma de los elementos de dicho vector.
 
// Autor: Antonio Vélez Estévez
// Fecha: 01/07/2016
 
Algoritmo FuncionesPasoReferencia01

//Sección de definición de variables constantes
const 
	MAX = 30
//Sección de definición de tipos
tipo
	vector [MAX] de entero: Vector
//Sección de declaración de variables globales	
var

//Sección de definición de subalgoritmos: funciones y procedimientos	
real funcion operacionesVector(E Vector: vect, E/S real: suma)
var
	entero: i
inicio
	suma <- 0

	desde i <- 0 hasta MAX - 1 hacer
		suma <- suma + vect[i]
	fin_desde
	
	devolver suma div MAX
fin_funcion

//Comienzo del algoritmo (Obligatorio)
principal	

var
	entero: i
	Vector: vect
	real: media
	real: suma
inicio 	
	desde i <- 0 hasta MAX - 1 hacer
		vect[i] <- i
	fin_desde
	
	media <- operacionesVector(vect, suma)
	
	escribir("La media es ", media)
	escribir("La suma es", suma)
fin_principal
fin_algoritmo
\end{lstlisting}
\subsection{Procedimientos}
\subsubsection{Paso por valor}
\begin{lstlisting}[language = pseudocodigoesp]
  
// Algoritmo: Ejemplo de procedimiento con paso por valor.
// Autor: Antonio Vélez Estévez
// Fecha: 30/06/2016
 
Algoritmo ProcedimientosPasoReferencia01

//Sección de definición de variables constantes
const 
//Sección de definición de tipos
tipo
//Sección de declaración de variables globales	
var

//Sección de definición de subalgoritmos: funciones y procedimientos	
procedimiento dimeNumero(E entero: x)

inicio 

	segun_sea(x) hacer
		caso 1: escribir("Su numero es el uno.")
		caso 2: escribir("Su numero es el dos.")
		caso 3: escribir("Su numero es el tres.")
		caso 4: escribir("Su numero es el cuatro.")
		caso 5: escribir("Su numero es el cinco.")
		caso 6: escribir("Su numero es el seis.")
		caso 7: escribir("Su numero es el siete.")
		caso 8: escribir("Su numero es el ocho.")
		caso 9: escribir("Su numero es el nueve.")
		caso 0: escribir("Su numero es el cero.")
		en_otro_caso: escribir("No se cual es su numero, probablemente
                haya introducido algo fuera del rango establecido.")
	fin_segun

fin_procedimiento
//Comienzo del algoritmo (Obligatorio)
principal	

var
	 entero: numero
inicio 	
         escribir("Introduzca un número entre 0 y 9 y le diré con
         letras qué numero es:")
	 leer(numero)
	 dimeNumero(numero)
fin_principal
fin_algoritmo
\end{lstlisting}
\subsubsection{Paso por referencia}
\begin{lstlisting}[language = pseudocodigoesp]
// Algoritmo: Ejemplo de procedimiento con paso por referencia
// Autor: Antonio Vélez Estévez
// Fecha: 30/06/2016

 
Algoritmo ProcedimientosPasoReferencia01

//Sección de definición de variables constantes
const 
//Sección de definición de tipos
tipo
//Sección de declaración de variables globales	
var

//Sección de definición de subalgoritmos: funciones y procedimientos	
procedimiento intercambiar(E/S real: x, E/S real: z)
var
	real: aux
inicio 
	aux <- x
	x <- z
	z <- aux
fin_procedimiento
//Comienzo del algoritmo (Obligatorio)
principal	

var
	real: numero1
	real: numero2
inicio 	
	numero1 <- 5
	numero2 <- 8
	escribir("La variable numero1 tiene el valor: ", numero1)
	escribir("La variable numero2 tiene el valor: ", numero2)
	
	escribir("Intercambiando...")
	
	intercambiar(numero1, numero2)
	
	escribir("La variable numero1 tiene el valor: ", numero1)
	escribir("La variable numero2 tiene el valor: ", numero2)
fin_principal
fin_algoritmo
\end{lstlisting}
\subsection{Registros}
\begin{lstlisting}[language = pseudocodigoesp]
// Algoritmo: Ejemplo de registro.
// Autor: Antonio Vélez Estévez
// Fecha: 01/07/2016
 
Algoritmo Registros01

//Sección de definición de variables constantes
const 
//Sección de definición de tipos
tipo
	registro: Persona
		cadena: nombre
		cadena: apellido
		entero: edad
		entero: NIF
	fin_registro
//Sección de declaración de variables globales	
var

//Sección de definición de subalgoritmos: funciones y procedimientos	

//Comienzo del algoritmo (Obligatorio)
principal	

var
	Persona: per
inicio 	
	escribir("Introduzca el nombre de la persona(máx. 30 caracteres):")
	leer(per.nombre)
	
	escribir("Introduzca el apellido de la persona(máx. 30 caracteres):")
	leer(per.apellido)
	
	escribir("Introduzca la edad de la persona:")
	leer(per.edad)
	
	escribir("Introduzca el NIF de la persona:")
	leer(per.NIF)
	
	escribir("Su persona es:")
	escribir("Nombre: ", per.nombre)
	escribir("Apellido: ", per.apellido)
	escribir("Edad: ", per.edad)
	escribir("NIF: ", per.NIF)
	
	
fin_principal
fin_algoritmo
\end{lstlisting}
\subsection{Registros anidados}
\begin{lstlisting}[language = pseudocodigoesp]
/*
 * Algoritmo: Ejemplo de registros anidados.
 * Autor: Antonio Vélez Estévez
 * Fecha: 01/07/2016
 */
 
Algoritmo RegistrosAnidados01

//Sección de definición de variables constantes
const 
//Sección de definición de tipos
tipo
	registro: Coche
		cadena: marca
		entero: numBastidor
		cadena: matricula
	fin_registro
	
	registro: Persona
		cadena: nombre
		cadena: apellido
		entero: edad
		entero: NIF
		Coche: coche
	fin_registro
//Sección de declaración de variables globales	
var

//Sección de definición de subalgoritmos: funciones y procedimientos	

//Comienzo del algoritmo (Obligatorio)
principal	

var
	Persona: per
inicio 	
	escribir("Introduzca el nombre de la persona(máx. 30 caracteres):")
	leer(per.nombre)
	
	escribir("Introduzca el apellido de la persona(máx. 30 caracteres):")
	leer(per.apellido)
	
	escribir("Introduzca la edad de la persona:")
	leer(per.edad)
	
	escribir("Introduzca el NIF de la persona:")
	leer(per.NIF)
	
	escribir("Introduzca la marca del coche de la persona:")
	leer(per.coche.marca)
	
	escribir("Introduzca el número de bastidor del coche de la persona:")
	leer(per.coche.numBastidor)
	
	escribir("Introduzca la matrícula del coche:")
	leer(per.coche.matricula)
	
	escribir("Su persona es:")
	escribir("Nombre: ", per.nombre)
	escribir("Apellido: ", per.apellido)
	escribir("Edad: ", per.edad)
	escribir("NIF: ", per.NIF)
	escribir("Además esta persona tiene un coche cuyos datos son:")
	escribir("Marca: ", per.coche.marca)
	escribir("Número de bastidor: ", per.coche.numBastidor)
	escribir("Matrícula: ", per.coche.matricula)
	
fin_principal
fin_algoritmo
\end{lstlisting}
\subsection{Matrices}
\begin{lstlisting}[language = pseudocodigoesp]
/*
 * Algoritmo: Ejemplo de matrices. Comprueba si una matriz es la identidad.
 * Autor: Antonio Vélez Estévez
 * Fecha: 01/07/2016
 */
 
Algoritmo Matrices01

//Sección de definición de variables constantes
const 
	MAX = 3
//Sección de definición de tipos
tipo
	matriz [MAX][MAX] de entero: Matriz
var

//Sección de definición de subalgoritmos: funciones y procedimientos	

//Comienzo del algoritmo (Obligatorio)
principal	

var
	Matriz : mat
	entero : i
	entero : j
	logico : esIdentidad
inicio 	
        escribir("A continuacion introducira los elementos de la matriz para comprobar
        si es identidad")
	desde i <- 0 hasta MAX - 1 hacer
		desde j <- 0 hasta MAX - 1 hacer
			escribir("Introduzca el elemento ", i, ", ", j)
			leer(mat[i][j])
		fin_desde
	fin_desde

	esIdentidad <- verdadero
	
	i <- 0
	mientras(i < MAX y esIdentidad) hacer
		j <- 0
		mientras(j < MAX y esIdentidad) hacer
		        // Si estamos en la diagonal el elemento debe ser uno si no,
                        // no será identidad.
			si(i = j) entonces
				si(mat[i][j] != 1) entonces
					esIdentidad <- falso
				fin_si
			// Si no estamos en la diagonal el elemento debe ser cero si
			// no, no será identidad.
			si_no
				si(mat[i][j] != 0) entonces
					esIdentidad <- falso
				fin_si
			fin_si		
			j <- j + 1
		fin_mientras
		i <- i + 1
	fin_mientras
	
	si(esIdentidad = verdadero) entonces
		escribir("La matriz introducida era la identidad.")
	si_no
		escribir("La matriz introducida no era la identidad.")
	fin_si
	
fin_principal
fin_algoritmo
\end{lstlisting}
\subsection{Vectores}
  \begin{lstlisting}[language = pseudocodigoesp]
/*
 * Algoritmo: Ejemplo de vectores.
 * Autor: Antonio Vélez Estévez
 * Fecha: 01/07/2016
 */
 
Algoritmo Vectores01

//Sección de definición de variables constantes
const 
	MAX = 10
//Sección de definición de tipos
tipo
	vector [MAX] de entero: Vector
var

//Sección de definición de subalgoritmos: funciones y procedimientos	

//Comienzo del algoritmo (Obligatorio)
principal	

var
	Vector : numeros
	entero : i
	real   : suma
	real   : media
inicio 	
        escribir("A continuación ingresará una serie de numeros para calcular la
        media y la suma de los mismos")
	desde i <- 0 hasta MAX - 1 hacer
		escribir("Ingrese el elemento ", i)
		leer(numeros[i])
		suma <- suma + numeros[i]
	fin_desde
	
	media <- suma div MAX
	
	escribir("La media del vector introducido es: ", media)
	escribir("La suma del vector introducido es:", suma)
	
fin_principal
fin_algoritmo
  \end{lstlisting}
\end{appendices}
\end{document}

